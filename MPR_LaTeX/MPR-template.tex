% The 8th Joint Workshop on Machine Perception and Robotics
% --MPR 2012 Fukaoka, Japan, Oct. 16-17, 2012

% This file is edited based on the formatting of IEEE Conference Proceedings

% Edit it to your own paper. You can modify this file and MPR-head.tex

\input{MPR-head.tex}

\begin{document}

% paper title
\title{Semantic Segmentation of 3D LiDAR Data at Dynamic Urban Scenes}


% avoiding spaces at the end of the author lines is not a problem with
% conference papers because we don't use \thanks or \IEEEmembership


\author{\authorblockN
	{Biao Gao\authorrefmark{1},
		Jilin Mei\authorrefmark{1}, 
		Donghao Xu\authorrefmark{1},
		Xijun Zhao\authorrefmark{2},
		Wen Yao\authorrefmark{2},
		Huijing Zhao\authorrefmark{1}}
\authorblockA{\authorrefmark{1}Peking University, Beijing, China}
\authorblockA{\authorrefmark{2}201 Institution, Beijing, China}}


% use only for invited papers
%\specialpapernotice{(Invited Paper)}

% make the title area
\maketitle

\begin{abstract}
This work studies semantic segmentation of 3D LiDAR data at dynamic urban scenes. LiDAR data plays an important role of perception in autonomous driving system. However, most semantic segmentation methods and datasets are designed for camera data nowadays. In this work, we propose a method which can generate semantic segmentation of LiDAR data and we evaluate its performance on a new 3D point cloud dataset collected in dynamic urban scenes by our driving platform. The experiments show that our method can recognize more kinds of labels and achieve an impressive result in dynamic urban scenes.
\end{abstract}

% no key words

\section{Introduction}
% no \PARstart
empty

\section{Related works}
% no \PARstart
empty

\section{Methodology}

\subsection{Data Preprocessing}
The point cloud data for LiDAR is sparse and unorganized, so it is time-consuming to find neighboring relations between different points. In order to process these unorganized point cloud data with deep convolutional neural network, we convert the point cloud data into 2D range image by cylindrical projection. After that, it will be easier to implement deep convolution neural network on the LiDAR data.

After cylindrical projection, the point cloud data will be encoded as a dense matrix with shape of $[H,W,C]$. $H$ means the number of lines for the specific LiDAR sensor (such as $H=32$ for Velodyne HDL-32E). $W$ equals to the number of points within each LiDAR scan line. $C$ is the channels' number in the range image. Here, $C$ is set to 3, which represents $[Range,Intensity,Height]$ three channels.

For each point $p_k=\langle x_k,y_k,z_k\rangle$ from the raw point cloud set $P$, the value of $Range$ in range image $R$ is defined as $r_k$:
 
\begin{equation}
r_k=\sqrt{{x_k}^2+{y_k}^2+{z_k}^2},  r_k\subset [0,255]	
\end{equation} 

Similarly, the values of $Intensity$ and $Height$ are normalized into $[0,255]$. These channels are very important properties of LiDAR data, which are enough to describe various objects in dynamic urban scenes.
 
\subsection{Problem Definition}

Let $X$ denotes the range image extracted by cylindrical projection of 3D point cloud data $S$. In this kind of projection, there is a one-to-one correspondence between a 3D point in one frame point cloud data and a pixel in the range image $X$. As a result, the semantic segmentation task of 3D point cloud data is equal with giving each pixel $x$ in the range image $X$ a label $y$. The problem of this work is formulated as learning a semantic segmentation model $f_{\theta}$ which maps each pixel $x$ to a label $y\in\{1,...,K\}$, and subsequently associate $y$ to the 3D points of $S$.

\begin{equation}
f_{\theta}: x\to y \in \{1,...,K\}
\end{equation}

The data samples are in the form of range images $X$. Given a set of supervised data samples $X_l=\{x_i, y_i\}$, where $\{x_i\}$ traverses each pixel of $X$ and $\{y_i\}$ are labels for $\{x_i\}$, annotated manually by human annotators. 
In order to learning a semantic segmentation model $f_\theta$, we need to find the best parameter set $\theta^*$ that minimize a loss function $L$ as below.

\begin{equation}
\theta^{*}=\mathop{\arg\max}_{\theta}L(X_l; \theta)
\end{equation}

\subsection{Network Architecture and Loss Function}
We use a FCN (Fully Convolutional Network) architecture for this semantic segmentation task.
Compared with common deep convolutional networks, it removes last fully connected layers, and replaces them with the in-network up-sampled or de-convolutional predictions of convolutional layers as predicted feature maps. During training procedure, it generally computes cross-entropy like losses in pixel-wise, between the predicted labels and ground truths.

Our network is trained via end-to-end guided by the designed loss function. Because the number of pixels are imbalanced between different classes and a lot of invalid or unknown-class pixels, the following multi-class weighted cross entropy loss function is designed to regular the weights between imbalanced classes.

For labeled data $X_l$, we implement some changes on the widely-used definition of cross entropy, and define loss function $L_l$ as below:

\begin{equation}
\begin{split}
&\varGamma_{i,j}=\{
	\begin{array}{lr}
	\overrightarrow{\varphi_k} ,\quad\quad if \quad [y_{i,j}\neq k \quad and\quad y_{i,j}\neq0]	\\
	\overrightarrow{0} ,\quad\quad\quad\quad\quad\quad  otherwise
	\end{array}	\\
	\\
&\textit{}L_l(X_l,Y_l;\theta)=-\frac{1}{H*W}\sum_{i=0}^{H-1}\sum_{j=0}^{W-1}\sum_{k=0}^K{\varGamma_{i,j}\omega_{k}ln(P^k_{\theta}(x_{i,j}))}
\end{split}
\end{equation}
Where $\varGamma_{i,j}$ is a one-hot vector $\varphi_k$ of label $k$, if $y_{i,j}\neq k$ and $y_{i,j}\neq0$. Label 0 means invalid or unknown pixels, including many fine fragments belong to background or hard to be annotated, so we don't want to evaluate these pixels if they are predicted as non-zero labels. $\omega_{k}$ here is used to balance the sample numbers between different labels and $P^k_{\theta}(x_{i,j})$ is the probability that pixel $x_{i,j}$ be assigned a label $k$ by our semantic segmentation model with the set of parameters $\theta$.


\section{Experiment}
\subsection{Data Set}
The performance of the proposed method is evaluated on a dynamic campus data set collected by an instrumented vehicle, which has a GPS/IMU suite and a Velodyne-HDL32, as shown in Fig. \ref{fig:collect_route}. The total route contains 1375 LiDAR frames. 880 frames for training, 220 frames for validation and 275 frames for testing.

\begin{figure}[ht]
\centering
\includegraphics[width=1.0\linewidth]{fig/collect_route}
% 这个图要更新!!!!!!!!!
\caption{The routes of data collection and the platform configuration.}
\label{fig:collect_route}
\end{figure}

\begin{table}[hb]
	\begin{center} \caption{Categories Distribution in Dataset}
		\label{category_distribution}
		\renewcommand{\arraystretch}{1.3}
		\begin{tabular}{|c|c|c|c|c|c|}
			\hline
			 - & People & Car & Vegetation & Building & Road	\\
			\hline
			Pixels & 733,664 & 257,239 & 4,661,880 & 6,579,360 & 16,856,614	\\
			\hline
		\end{tabular}
	\end{center}
\end{table}

High quality pixel-level annotation is necessary for network training. Instead of working on the raw point cloud, human annotators work on the range image where object regions are associated with the ones in adjacent frames. Annotators only need to assign the category of some region in one frame, then a series of associated regions are marked with the same label. Although sometimes the data association brings errors, it largely reduce the annotation time. 

The categories distribution in this dataset is shown in TABLE . \ref{category_distribution}. Obviously, the data distribution is imbalanced between categories. So, we apply each label an unique weight based on data distribution to reduce the influence of data imbalance.

\subsection{Setup}
Our method is implemented with a FCN (Fully Convolutional Network). The range image size is 1080x32, which width is down-sampled for efficiency. A small batch size for training sets will be better. The network is implemented with TensorFlow in the environment with NVIDIA TITAN X GPU. We use the AdamOptimizer with 1e-5 learning rate.

It's important to aware that our data frames are captured sequentially. In order to avoid data correlation between adjacent frames, shuffling them before training process is necessary.

\subsection{Preparing your Electronic Paper}

{\em Type sizes and typefaces}: Follow the type sizes specified in
Table~\ref{table1}. As an aid in gauging type size, 1 point is
about 0.35~mm. The size of the lowercase letter ``j'' will give
the point size. Times New Roman is the preferred font.

{\em 1) US Letter Margins}:  top = 0.75 inches, bottom = 1 inch, side = 0.625 inches. 
Each column measures 3.5 inches wide, with a 0.25-inch measurement between columns.

{\em 2) A4 Margins}: top = 19mm, bottom = 43mm, side = 13 mm. The A4 column width is 
88mm (3.45 in). The space between the two columns is 4mm (0.17 in). Paragraph indentation 
is 3.5 mm (0.14 in).

\begin{table}[hb]
\begin{center} \caption{Type Sizes for Papers}
\label{table1}
\renewcommand{\arraystretch}{1.3}
\begin{tabular}{|c|p{34mm}|c|c|}
 \hline
Type & \multicolumn{3}{c|}{Appearance} \\
size &\multicolumn{3}{c|}{~} \\ \cline{2-4} (pts.) & Regular &
Bold & Italic \\ \hline
 6 & Table captions\footnotemark[1], table superscripts & & \\ \hline
 8 & Section titles\footnotemark[1], references, tables, table
 names\footnotemark[1],
 first letters in table captions\footnotemark[1], figure captions, footnotes,
 text subscripts, and superscripts &
& \\ \hline
 9 & & Abstract & \\ \hline
 10 & Authors' affiliations, main text, equations, first letters in
 section titles\footnotemark[1] & & Subheading \\ \hline
 11 & Authors' names & & \\ \hline
 24 & Paper title & & \\ \hline
\end{tabular}
\end{center}
\end{table}
\footnotetext[1]{Uppercase    ( {\bf It is recommended that footnotes be avoided. Instead, try to integrate the footnote information into the text.} )}

The column width is 82~mm (3.23 in). The space between the two columns is
6~mm (0.24 in). Paragraph indentation is 3.5 mm (0.14 in).

Left- and right-justify your columns. Use tables and figures to
adjust column length. On the last page of your paper, adjust the
lengths of the columns so that they are equal. Use automatic
hyphenation and check spelling.


\section{Helpful Hints}

\subsection{Figures and Tables - Subsection Example}

Position figures and tables at the tops and bottoms of columns. Avoid
placing them in the middle of columns. Large figures and tables may span
across both columns. Figure captions should be centered below the figures;
table captions should be centered above. Avoid placing figures and tables
before their first mention in the text. Use the abbreviation ``Fig. 1'',
even at the beginning of a sentence.

\subsubsection{Sub-subsection example}
Figure axis labels are often a source of confusion. Use words rather than
symbols. For example, write ``Magnetization'', or ``Magnetization, M'',
not just ``M.''  Put units in parentheses. Do not label axes only with
units. In the example, write ``Magnetization (A/m)'' or ``Magnetization (A
$\cdot$ m$^{-1}$).'' Do not label axes with a ratio of quantities and
units. For example, write ``Temperature (K)'', not ``Temperature/K.''

Multipliers can be especially confusing. Write ``Magnetization (kA/m)'' or
``Magnetization ($10^{3}$ A/m)''. Figure labels should be legible, about
10-point type.


\subsection{References}

Number citations consecutively in square brackets \cite{eason}.
Punctuation follows the bracket \cite{maxwell}. Refer simply to the
reference number, as in \cite{jacobs}. Use ``Ref. \cite{jacobs}'' or
``Reference \cite{jacobs}'' at the beginning of a sentence: ``Reference
\cite{jacobs} was the first ...''

Number footnotes separately in superscripts. Place the actual footnote at
the bottom of the column in which it was cited. Do not put footnotes in
the reference list. Use letters for table footnotes (see Table 1). IEEE
Transactions no longer use a journal prefix before the volume number. For
example, use ``IEEE Trans. Magn., vol. 25'', not ``vol. MAG-25''.

Give all authors' names; use ``et al.'' if there are six authors or more.
Papers that have not been published, even if they have been submitted for
publication, should be cited as "unpublished" \cite{elissa}. Papers that
have been accepted for publication should be cited as ``in press''
\cite{nicole}. In a paper title, capitalize the first word and all other
words except for conjunctions, prepositions less than seven letters, and
prepositional phrases.

\begin{figure}[t]
 \centering
% \includegraphics[scale=0.95] {fig1}
\caption{Magnetization as a function of applied field.} \label{figure1}
\end{figure}

For papers published in translated journals, first give the English
citation, then the original foreign-language citation \cite{yorozu}.


\subsection{Abbreviations and Acronyms}

Define abbreviations and acronyms the first time they are used in the
text, even after they have been defined in the abstract. Abbreviations
such as IEEE, SI, MKS, CGS, sc, dc, and rms do not have to be defined. Do
not use abbreviations in the title unless they are unavoidable.

\subsection{Equations}

Number equations consecutively with equation numbers in
parentheses flush with the right margin, as in (\ref{eq1}). To
make your equations more compact, you may use the solidus (/), the
exp function, or appropriate exponents. Italicize Roman symbols
for quantities and variables, but not Greek symbols. Use an en
dash (-) rather than a hyphen for a minus sign. Use parentheses to
avoid ambiguities in denominators. Punctuate equations with commas
or periods when they are part of a sentence, as in
\begin{equation}
 z = \sin^2 x + \cos^2 y.  \label{eq1}
\end{equation}
Symbols in your equation should be defined before the equation
appears or immediately following. Use ``(\ref{eq1}),'' not
``Eq.~(\ref{eq1})'' or ``equation~(\ref{eq1}),'' except at the
beginning of a sentence: ``Equation~(\ref{eq1}) is ...''


\subsection{Other Recommendations}

The Roman numerals used to number the section headings are
optional. If you do use them, do not number ACKNOWLEDGMENTS and
REFERENCES, and begin Subheadings with letters. Use two spaces
after periods (full stops). Hyphenate complex modifiers:
``zero-field-cooled magnetization.'' Avoid dangling participles,
such as, ``Using (\ref{eq1}), the potential was calculated.''
Write instead, ``The potential was calculated using (\ref{eq1}),''
or ``Using  (\ref{eq1}), we calculated the potential.''

Use a zero before decimal points: ``0.25'', not ``.25''. Use ``cm$^{3}$''
not ``cc''. Do not mix complete spellings and abbreviations of units:
``Wb/m$^{2}$'' or ``webers per square meter,'' not ``webers/m$^{2}$''.
Spell units when they appear in text: ``...a few henries'', not ``...a few
H.'' If your native language is not English, try to get a native
English-speaking colleague to proofread your paper. Do not add page
numbers.

\section{Units}

Use either SI (MKS) or CGS as primary units. (SI units are encouraged.)
English units may be used as secondary units (in parentheses). An
exception would be the use of English units as identifiers in trade, such
as ``3.5-inch disk drive.''

Avoid combining SI and CGS units, such as current in amperes and magnetic
field in oersteds. This often leads to confusion because equations do not
balance dimensionally.


\section{Some Common Mistakes}

The word ``data'' is plural, not singular. The subscript for the
permeability of vacuum is zero, not a lowercase letter ``o''. In American
English, periods and commas are within quotation marks, like ``this
period''. A parenthetical statement at the end of a sentence is punctuated
outside of the closing parenthesis (like this). (A parenthetical sentence
is punctuated within the parentheses.) A graph within a graph is an
``inset'', not an ``insert''. The word alternatively is preferred to the
word ``alternately'' (unless you mean something that alternates). Do not
use the word ``essentially'' to mean ``approximately'' or ``effectively.''
Be aware of the different meanings of the homophones  ``affect'' and
``effect,'' ``complement'' and ``compliment,'' ``discreet'' and
``discrete,'' ``principal'' and ``principle.'' Do not confuse ``imply''
and ``infer.'' The prefix ``non'' is not a word; it should be joined to
the word it modifies, usually without a hyphen. There is no period after
the ``et'' in the Latin abbreviation ``et al.'' The abbreviation ``i.e.''
means ``that is,'' and the abbreviation ``e.g.'' means ``for example.'' An
excellent style manual for science writers is \cite{young}.



% Reminder: the "draftcls" or "draftclsnofoot", not "draft", class option
% should be used if it is desired that the figures are to be displayed while
% in draft mode.

% An example of a floating figure using the graphicx package.
% Note that \label must occur AFTER (or within) \caption.
% For figures, \caption should occur after the \includegraphics.
%
%\begin{figure}
%\centering
%\includegraphics[width=2.5in]{myfigure}
% where an .eps filename suffix will be assumed under latex, 
% and a .pdf suffix will be assumed for pdflatex
%\caption{Simulation Results}
%\label{fig_sim}
%\end{figure}


% An example of a double column floating figure using two subfigures.
%(The subfigure.sty package must be loaded for this to work.)
% The subfigure \label commands are set within each subfigure command, the
% \label for the overall fgure must come after \caption.
% \hfil must be used as a separator to get equal spacing
%
%\begin{figure*}
%\centerline{\subfigure[Case I]{\includegraphics[width=2.5in]{subfigcase1}
% where an .eps filename suffix will be assumed under latex, 
% and a .pdf suffix will be assumed for pdflatex
%\label{fig_first_case}}
%\hfil
%\subfigure[Case II]{\includegraphics[width=2.5in]{subfigcase2}
% where an .eps filename suffix will be assumed under latex, 
% and a .pdf suffix will be assumed for pdflatex
%\label{fig_second_case}}}
%\caption{Simulation results}
%\label{fig_sim}
%\end{figure*}



% An example of a floating table. Note that, for IEEE style tables, the 
% \caption command should come BEFORE the table. Table text will default to
% \footnotesize as IEEE normally uses this smaller font for tables.
% The \label must come after \caption as always.
%
%\begin{table}
%% increase table row spacing, adjust to taste
%\renewcommand{\arraystretch}{1.3}
%\caption{An Example of a Table}
%\label{table_example}
%\begin{center}
%% Some packages, such as MDW tools, offer better commands for making tables
%% than the plain LaTeX2e tabular which is used here.
%\begin{tabular}{|c||c|}
%\hline
%One & Two\\
%\hline
%Three & Four\\
%\hline
%\end{tabular}
%\end{center}
%\end{table}


%\section{Conclusion}
%The conclusion goes here.

% conference papers do not normally have an appendix

% use section* for acknowledgement
\section*{Acknowledgment}
% optional entry into table of contents (if used)
%\addcontentsline{toc}{section}{Acknowledgment}

The preferred spelling of the word "acknowledgment" in America is without
an "e" after the "g". Try to avoid the stilted expression, "One of us
(R.B.G.) thanks ...". Instead, try "R.B.G. thanks ...". Put sponsor
acknowledgments in the unnumbered footnote on the first page.

% trigger a \newpage just before the given reference
% number - used to balance the columns on the last page
% adjust value as needed - may need to be readjusted if
% the document is modified later
%\IEEEtriggeratref{8}
% The "triggered" command can be changed if desired:
%\IEEEtriggercmd{\enlargethispage{-5in}}

% references section
% NOTE: BibTeX documentation can be easily obtained at:
% http://www.ctan.org/tex-archive/biblio/bibtex/contrib/doc/

% can use a bibliography generated by BibTeX as a .bbl file
% standard IEEE bibliography style from:
% http://www.ctan.org/tex-archive/macros/latex/contrib/supported/IEEEtran/testflow/bibtex
%\bibliographystyle{IEEEtran.bst}
% argument is your BibTeX string definitions and bibliography database(s)
%\bibliography{IEEEabrv,../bib/paper}
%
% <OR> manually copy in the resultant .bbl file
% set second argument of \begin to the number of references
% (used to reserve space for the reference number labels box)

\begin{thebibliography}{10}

\bibitem{eason} G. Eason, B. Noble, and I. N. Sneddon, ``On certain integrals of
Lipschitz-Hankel type involving products of Bessel functions,'' {\em Phil.
Trans. Roy. Soc. London}, vol. A247, pp. 529--551, April 1955.

\bibitem{maxwell} J. Clerk Maxwell, {\em A Treatise on Electricity and Magnetism}, 3rd ed.,
vol. 2. Oxford: Clarendon, 1892, pp.68--73.

\bibitem{jacobs} I. S. Jacobs and C. P. Bean, ``Fine particles, thin films and exchange
anisotropy,'' in {\em Magnetism}, vol. III, G. T. Rado and H. Suhl, Eds.
New York: Academic, 1963, pp. 271--350.

\bibitem{elissa} K. Elissa, ``Title of paper if known,'' unpublished.

\bibitem{nicole} R. Nicole, ``Title of paper with only first word capitalized'',
{\em J. Name Stand. Abbrev.}, in press.

\bibitem{yorozu} Y. Yorozu, M. Hirano, K. Oka, and Y. Tagawa, ``Electron spectroscopy
studies on magneto-optical media and plastic substrate interface,'' {\em
IEEE Transl. J. Magn. Japan}, vol. 2, pp. 740--741, August 1987 [Digests
9th Annual Conf. Magnetics Japan, p. 301, 1982].

\bibitem{young} M. Young, {\em The Technical Writer's Handbook}. Mill Valley, CA: University
Science, 1989.


\end{thebibliography}


% that's all folks
\end{document}
